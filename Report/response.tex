\documentclass{article}
\title{Response}


\begin{document}
\maketitle
\section{Response to the reviewers}
We have placed the reviewers comments in \textit{\textbf{Italics and Bold}} and our response to those comments in plain text.

\subsection{Review 1 (Reviewer A)}
\paragraph*{\textit{Introduction: The introduction is well written but oddly doesn't contain discussion about cooperative games. The introduction starts by discussing general game playing (which is not a main topic of the paper) and then presents MCTS (which is) but the discussion of cooperative games is missing.}}

\paragraph*{\textit{When listing objects in  II.A the default gray empty square should also be mentioned.}}
We have added a definition for the \emph{Floor}.
\paragraph*{\textit{In the definition of 'Walls' the word 'Walls' appears instead of 'Doors'.}}
We have corrected the typo in the definition of \emph{Walls} in the problem domain section.
\paragraph*{\textit{When specifying that closed 'Doors' are blue there should be mention of the color of open 'Doors' (I assume gray).}}
We have changed the description of \emph{Doors} to better reflect what they look like when open.
\paragraph*{\textit{It is unclear if 'Door' closes again when agent on button moves on (or, alternatively, if walking onto a 'Button' effectively destroy the 'Door').}}
We have improved the definition for \emph{Door} in order to make it clearer that the \emph{Door} closes again when the \emph{Buttons} are no longer activated.
\paragraph*{\textit{Since 'Agents' are described as dark yellow, and since there are only two shades of yellow used, it would be wiser to describe a 'Goal' as rendered in bright yellow.}}
We have reworded the definition of \emph{Goal} to better describe the colour in relation to \emph{Agents}.
\paragraph*{\textit{the sentence "Agent 1 reached an item like this -Agent 2 would seemingly gain some purpose and typically head for the other items." is poorly worded and not very informative. It is better explained in the sentences below. Either remove this sentence or reword it.}}

\paragraph*{\textit{it would be a good idea to give the maps meaningful names.}}
\paragraph*{\textit{The MAGA is not explained very well and seems hastily put together. I would say explain this better but given how poorly it does and how badly optimized it seems to be, maybe the authors should get rid of this player altogether. If this work is resubmitted in future I would recommend creating a better thought out version of this player. For now the better VMAGA should stand alone.
}}
\paragraph*{\textit{There should be a separate subsection for the results.}}
\paragraph*{\textit{In figures 4--9 the 3 MCTS players are ordered in an odd order Meduin-Large-Small. This order makes no sense and should be changed. GA-Controller is presented separately from Var-GA inexplicably. If this player is kept (and it shouldn't be. Unless changed and improved for submission somewhere else) then both GA players should be together. It seems that columns are organized automatically in alphabetical order they should be ordered in a way that makes sense.}}
\paragraph*{\textit{Regarding the fact that players were trained assuming a random strategy was controlling the other player: This is mentioned again in the section, but a reasoning for this decision is not given. Why not use self play? Wouldn't it do better? Doesn't it make sense to assume a better than random strategy for the other player?}}

\subsection{Review 2 (Reviewer D)}
\paragraph*{\textit{The writing is generally good, I only noticed one actual error "GGP[3]", but the tone is casual and a lot of inappropriately subjective comments are made}}
\paragraph*{\textit{The authors make a few dubious claims in the Introduction. GGP is not exactly "the field of writing Artificial Intelligence (AI) agents", and it is wrong to say that "MCTS doesn't requuire any knowledge about the game itself in order to play", as it needs to know starting position, legal moves, terminal states, etc. Perhaps the authors means "strategic knowledge". }}
Have clarified the definition of GGP with reference to this paper\cite{genesereth2005general} and the terminology used to define MCTS use of knowledge. 
\paragraph*{\textit{The authors claim that MCTS works best when it is given rewards frequently. What is the evidence for this claim?}}
Have rewritten this to point out that MCTS can't provide anything but random if it isn't given enough time or depth to search to a state that provides a reward.
\paragraph*{\textit{The research problem is stated, but should be stated more clearly in the Introduction.}}
Added at the beginning of the introduction a clear statement of the research problem.
\paragraph*{\textit{"MCTS can charge off achieving its own goals": it seems odd to describe an algorithm as "charging off".}}
Reworded.
\paragraph*{\textit{"This proves that the challenge...": no it doesn't, it indicates it.}}
Reworded to indicates.
\paragraph*{\textit{"The shockingly good results...": too casual and subjective, overstating the case.}}
Reworded. 
\paragraph*{\textit{There is no Discussion as such, rather there are subjective appraisals of the algorithm's performance at each task scattered throughout the results. The paper would be better structures with a Discussion section that summarised the results and their implications, and then recapped this in the Conclusion. At the moment, the exact contributions are a bit hard to see.}}
\paragraph*{\textit{The list of references is short (four) and does not include any references to work on cooperative agents, swarm behaviour, etc.}}
\subsection{Review 3 (Reviewer G)}
\paragraph*{\textit{Who created this problem domain?}}
We have clarified who created the problem domain.
\paragraph*{\textit{The agents move simultaneously or alternately?}}
Added a section to highlight this.
\paragraph*{\textit{Do agents see all areas of the problem?}}
We have clarified how the problem domain handles observability and what information the \emph{Agents} have access to.
\paragraph*{\textit{If an agent reaches a goal, is it known to the other agents?}}
We have clarified whether the \emph{Agents} are aware that other \emph{Agents} reach a \emph{Goal}.

\paragraph*{\textit{In what sense this domain interesting compared to other domains?}}


\paragraph*{\textit{Is the problem domain a real time problem? How long is a tick? 40ms? Do all agents return an action within a tick? Possible 5 actions mean up, down left, right and stay where it is?}}
The definition of the problem domain has been improved with a section describing the particular way that actions are collected and executed as well as a list of each possible action that can be chosen.

\paragraph*{\textit{What is "Uniform Cost applied to Trees"? Is it something similar to the famous "Upper Confidence bound applied to Trees"? If not, please cite something.}}
We have corrected the incorrect definition of UCT from Uniform Cost over Trees to Upper Confidence bound applied to Trees.
\paragraph*{\textit{About MCTS, what are "iteration", "UCT Border", and "rollout border"? My guess is iteration means the number of rollouts,
UCT border means the UCT tree search depth limit, and rollout border means the length of the sequence of rollout. Am I correct? These phrases are different from normal terminology used in MCTS.}}
Reworded to more used terminology.
\paragraph*{\textit{What is the reward of the rollout? Maybe the raw score?}}
\paragraph*{\textit{About the results, what is averaged?
How many runs are averaged?
All agents play in cooperation with other agents including the "keyboard AI"?
Is the keyboard AI controlled by human beings?
Can we assume it plays perfectly well?}}
\paragraph*{\textit{Please cite something about (1+1)-ES}}
\paragraph*{\textit{If all information is visible to the agents,
this problem domain seems not so interesting especially if
both agents are the same.}}
\paragraph*{\textit{I don't think GA is suitable for this domain.
Comparing against A* search with a simple evaluation function would be more interesting.}}
Referenced the design of the GA that shows validity in its theoretical to perform in real time general video game domains. The production of the \emph{VariGA} was to better adapt GA to the grid based world after observing the GA struggling to navigate. Whilst neither AI proved able to solve the puzzle element of the domain, this served to show differences between MCTS and the GA techniques. A* was considered to have too high a branching factor in this problem but was decided to be investigated in future work.

\paragraph*{\textit{"Doors are rendered in Black." -> "Walls are rendered in Black."}}
We have corrected the typo in the definition of \emph{Walls} in the problem domain section.
\subsection{Review 4 (Reviewer B)}
\paragraph*{\textit{there is some mixup with agents being labeled 0 and 1 in the figures, but 1 and 2 in the text.}}
That section has been removed due to another comment that we received and so the mixup is gone. 

\paragraph*{\textit{I did not understand the terms "UCT border" and "rollout border". Explain them. Are they related to "Max UCT" and "Max Rollout" in Fig. 2 ?}}
Both have been reworded to a more used set of terminology.
\paragraph*{\textit{What does 1+1 ES mean?}}
The paper has been expanded to include a brief description of the 1+1 ES technique.
\paragraph*{\textit{I did not quite understand how your MCTS works with multiple agents. Are you running independent simulations for each agent (I assume yes but it was not clear)? }}
Yes, each agent was running independent simulations. It has been made clearer in the paper that none of the agents have the ability to communicate with another agent, and that each agent is controlled by a single controller. 
\paragraph*{\textit{I would like to see results for mixed agent teams.}}
\paragraph*{\textit{The issue of communication, especially learning communication, is very important and I suggest you focus on this for future research.}}
\subsection{Review 5 (Reviewer F)}
\paragraph*{\textit{The early segments fail to really discuss the MCTS algorithm.  The abbreviation for this is also found in the abstract and should really be in the introductory paragraph.}}
\paragraph*{\textit{Section II covering preliminary tests is light on technical details and seem ill placed when you consider the subsequent studies later on in the paper.}}
We have removed this section to make way for more detailed results
\paragraph*{\textit{The third page could do with being structured better.  The charts are ill-placed, do not help the flow of the authors argument and need better captions.}}
\paragraph*{\textit{The paper would benefit from showing the actual levels used as part of the testing.  Given it makes it harder to understand the impact of later levels.  Figure 10 should be shown much earlier in the paper.}}
\paragraph*{\textit{The selection of controller types does strike me as odd.  With a random selected and a Variable MAGA that is introduced to resolve shortcoming of the other.  What about a simple 1+1 ES without the macro's?  Or the likes of an A* bot?  Could these not be implemented?}}
A simple 1+1 ES would not be able to find a good solution within the time limits of real time video games. Macro Actions has been proven to vastly improve the ability of GA's to solve complex real time problems such as in Perez et al \cite{perez2013rolling}. A* was considered to have too high a branching factor in this problem but was decided to be investigated in future work.
\paragraph*{\textit{More than once the idea of a language through which to communicate is discussed, but it has no real impact on this paper and should really be saved for when it is explored}}

\bibliographystyle{IEEEtran}
\bibliography{ceec}
\end{document}