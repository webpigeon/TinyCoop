\documentclass{article}
\title{Response}


\begin{document}
\maketitle
\section{Response to the reviewers}
We have placed the reviewers comments in \textit{\textbf{Italics and Bold}} and our response to those comments in plain text.

\subsection{Review 1 (Reviewer A)}
\paragraph*{\textit{When listing objects in  II.A the default gray empty square should also be mentioned.}}
We have added a definition for the \emph{Floor}.
\paragraph*{\textit{In the definition of 'Walls' the word 'Walls' appears instead of 'Doors'.}}
We have corrected the typo in the definition of \emph{Walls} in the problem domain section.
\paragraph*{\textit{When specifying that closed 'Doors' are blue there should be mention of the color of open 'Doors' (I assume gray).}}
We have changed the description of \emph{Doors} to better reflect what they look like when open.
\paragraph*{\textit{It is unclear if 'Door' closes again when agent on button moves on (or, alternatively, if walking onto a 'Button' effectively destroy the 'Door').}}
We have improved the definition for \emph{Door} in order to make it clearer that the \emph{Door} closes again when the \emph{Buttons} are no longer activated.
\paragraph*{\textit{Since 'Agents' are described as dark yellow, and since there are only two shades of yellow used, it would be wiser to describe a 'Goal' as rendered in bright yellow.}}
We have reworded the definition of \emph{Goal} to better describe the colour in relation to \emph{Agents}.
\subsection{Review 2 (Reviewer D)}
\subsection{Review 3 (Reviewer G)}
\paragraph*{\textit{What is "Uniform Cost applied to Trees"? Is it something similar to the famous "Upper Confidence bound applied to Trees"? If not, please cite something.}}
We have corrected the incorrect definition of UCT from Uniform Cost over Trees to Upper Confidence bound applied to Trees.
\paragraph*{\textit{Is the problem domain a real time problem? How long is a tick? 40ms? Do all agents return an action within a tick?}}
The definition of the problem domain has been improved with a section describing the particular way that actions are collected and executed as well as a list of each possible action that can be chosen.
\paragraph*{\textit{If an agent reaches a goal, is it known to the other agents?}}
We have clarified whether the \emph{Agents} are aware that other \emph{Agents} reach a \emph{Goal}.
\paragraph*{\textit{Do agents see all areas of the problem?}}
We have clarified how the problem domain handles observability and what information the \emph{Agents} have access to.
\paragraph*{\textit{Who created this problem domain?}}
We have clarified who created the problem domain.
\paragraph*{\textit{"Doors are rendered in Black." -> "Walls are rendered in Black."}}
We have corrected the typo in the definition of \emph{Walls} in the problem domain section.
\subsection{Review 4 (Reviewer B)}
\subsection{Review 5 (Reviewer F)}
\end{document}