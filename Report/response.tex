\documentclass{article}
\title{Response}


\begin{document}
\maketitle
\section{Response to the reviewers}
We have placed the reviewers comments in \textit{\textbf{Italics and Bold}} and our response to those comments in plain text.

\subsection{Review 1 (Reviewer A)}
\paragraph*{\textit{When listing objects in  II.A the default gray empty square should also be mentioned.}}
We have added a definition for the \emph{Floor}.
\paragraph*{\textit{In the definition of 'Walls' the word 'Walls' appears instead of 'Doors'.}}
We have corrected the typo in the definition of \emph{Walls} in the problem domain section.
\paragraph*{\textit{When specifying that closed 'Doors' are blue there should be mention of the color of open 'Doors' (I assume gray).}}
We have changed the description of \emph{Doors} to better reflect what they look like when open.
\paragraph*{\textit{It is unclear if 'Door' closes again when agent on button moves on (or, alternatively, if walking onto a 'Button' effectively destroy the 'Door').}}
We have improved the definition for \emph{Door} in order to make it clearer that the \emph{Door} closes again when the \emph{Buttons} are no longer activated.
\paragraph*{\textit{Since 'Agents' are described as dark yellow, and since there are only two shades of yellow used, it would be wiser to describe a 'Goal' as rendered in bright yellow.}}
We have reworded the definition of \emph{Goal} to better describe the colour in relation to \emph{Agents}.
\subsection{Review 2 (Reviewer D)}
\paragraph*{\textit{The writing is generally good, I only noticed one actual error "GGP[3]", but the tone is casual and a lot of inappropriately subjective comments are made}}
\paragraph*{\textit{The authors make a few dubious claims in the Introduction. GGP is not exactly "the field of writing Artificial Intelligence (AI) agents", and it is wrong to say that "MCTS doesn't requuire any knowledge about the game itself in order to play", as it needs to know starting position, legal moves, terminal states, etc. Perhaps the authors means "strategic knowledge". }}
Have clarified the definition of GGP with reference to this paper\cite{genesereth2005general} and the terminology used to define MCTS use of knowledge. 
\paragraph*{\textit{The authors claim that MCTS works best when it is given rewards frequently. What is the evidence for this claim?}}
Have rewritten this to point out that MCTS can't provide anything but random if it isn't given enough time or depth to search to a state that provides a reward.
\paragraph*{\textit{The research problem is stated, but should be stated more clearly in the Introduction.}}
\paragraph*{\textit{"MCTS can charge off achieving its own goals": it seems odd to describe an algorithm as "charging off".}}
\paragraph*{\textit{"This proves that the challenge...": no it doesn't, it indicates it.}}
Reworded to indicates.
\paragraph*{\textit{"The shockingly good results...": too casual and subjective, overstating the case.}}
Reworded. 
\paragraph*{\textit{There is no Discussion as such, rather there are subjective appraisals of the algorithm's performance at each task scattered throughout the results. The paper would be better structures with a Discussion section that summarised the results and their implications, and then recapped this in the Conclusion. At the moment, the exact contributions are a bit hard to see.}}
\paragraph*{\textit{The list of references is short (four) and does not include any references to work on cooperative agents, swarm behaviour, etc.}}
\paragraph*{\textit{Despite these concerns with the presentation, I liked the idea and the authors' approach to investigating the problem. The paper could be improved greatly with a little bit of work.}}
\subsection{Review 3 (Reviewer G)}
\paragraph*{\textit{What is "Uniform Cost applied to Trees"? Is it something similar to the famous "Upper Confidence bound applied to Trees"? If not, please cite something.}}
We have corrected the incorrect definition of UCT from Uniform Cost over Trees to Upper Confidence bound applied to Trees.
\paragraph*{\textit{Is the problem domain a real time problem? How long is a tick? 40ms? Do all agents return an action within a tick?}}
The definition of the problem domain has been improved with a section describing the particular way that actions are collected and executed as well as a list of each possible action that can be chosen.
\paragraph*{\textit{If an agent reaches a goal, is it known to the other agents?}}
We have clarified whether the \emph{Agents} are aware that other \emph{Agents} reach a \emph{Goal}.
\paragraph*{\textit{Do agents see all areas of the problem?}}
We have clarified how the problem domain handles observability and what information the \emph{Agents} have access to.
\paragraph*{\textit{Who created this problem domain?}}
We have clarified who created the problem domain.
\paragraph*{\textit{"Doors are rendered in Black." -> "Walls are rendered in Black."}}
We have corrected the typo in the definition of \emph{Walls} in the problem domain section.
\subsection{Review 4 (Reviewer B)}
\paragraph*{\textit{there is some mixup with agents being labeled 0 and 1 in the figures, but 1 and 2 in the text.}}

\paragraph*{\textit{I did not understand the terms "UCT border" and "rollout border". Explain them. Are they related to "Max UCT" and "Max Rollout" in Fig. 2 ?}}
\paragraph*{\textit{What does 1+1 ES mean?}}
\paragraph*{\textit{I did not quite understand how your MCTS works with multiple agents. Are you running independent simulations for each agent (I assume yes but it was not clear)? }}
\paragraph*{\textit{I would like to see results for mixed agent teams.}}
\paragraph*{\textit{The issue of communication, especially learning communication, is very important and I suggest you focus on this for future research.}}
\subsection{Review 5 (Reviewer F)}
\paragraph*{\textit{The early segments fail to really discuss the MCTS algorithm.  The abbreviation for this is also found in the abstract and should really be in the introductory paragraph.}}
\paragraph*{\textit{Section II covering preliminary tests is light on technical details and seem ill placed when you consider the subsequent studies later on in the paper.}}
We have removed this section to make way for more detailed results
\paragraph*{\textit{The third page could do with being structured better.  The charts are ill-placed, do not help the flow of the authors argument and need better captions.}}
\paragraph*{\textit{The paper would benefit from showing the actual levels used as part of the testing.  Given it makes it harder to understand the impact of later levels.  Figure 10 should be shown much earlier in the paper.}}
\paragraph*{\textit{The selection of controller types does strike me as odd.  With a random selected and a Variable MAGA that is introduced to resolve shortcoming of the other.  What about a simple 1+1 ES without the macro's?  Or the likes of an A* bot?  Could these not be implemented?}}
\paragraph*{\textit{More than once the idea of a language through which to communicate is discussed, but it has no real impact on this paper and should really be saved for when it is explored}}

\bibliographystyle{IEEEtran}
\bibliography{ceec}
\end{document}